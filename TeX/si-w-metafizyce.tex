%%%%%%%%%%%%%%%%%%%%%%%%%%%%%%%%%%%%%%%%%%%%%%%%%%%%%%%%%%%%%%%%%%%%%
% LaTeX Template: Project Titlepage Modified (v 0.1) by rcx
%
% Original Source: http://www.howtotex.com
% Date: February 2014
% 
% This is a title page template which be used for articles & reports.
% 
% This is the modified version of the original Latex template from
% aforementioned website.
% 
%%%%%%%%%%%%%%%%%%%%%%%%%%%%%%%%%%%%%%%%%%%%%%%%%%%%%%%%%%%%%%%%%%%%%%

\documentclass[12pt]{article}
\usepackage[a4paper]{geometry}
\usepackage[myheadings]{fullpage}
\usepackage{fancyhdr}
\usepackage{amsmath}
\usepackage{amsthm}
\usepackage{lastpage}
\usepackage{graphicx, wrapfig, subcaption, setspace, booktabs}
\usepackage[T1]{fontenc}
\usepackage{listings}
\usepackage[font=small, labelfont=bf]{caption}
\usepackage{fourier}
\usepackage{xcolor}
\usepackage[protrusion=true, expansion=true]{microtype}
\usepackage[english,polish]{babel}
\usepackage{sectsty}
\usepackage{url, lipsum}
\usepackage{polski}
\usepackage[utf8]{inputenc}
\usepackage{hyperref}
\hypersetup{
	colorlinks,
	citecolor=black,
	filecolor=black,
	linkcolor=black,
	urlcolor=black
}
\usepackage{multicol}
\usepackage{dirtytalk}
\usepackage{amsfonts}


\newcommand{\HRule}[1]{\rule{\linewidth}{#1}}
\onehalfspacing
\setcounter{tocdepth}{5}
\setcounter{secnumdepth}{5}

%-------------------------------------------------------------------------------
% HEADER & FOOTER
%-------------------------------------------------------------------------------
\pagestyle{fancy}
\fancyhf{}
\setlength\headheight{15pt}
\fancyhead[L]{Artur M. Brodzki}
\fancyhead[R]{\textit{Sztuczna inteligencja w metafizyce}, KPF 2018Z}
\fancyfoot[C]{\thepage}
%-------------------------------------------------------------------------------
% TITLE PAGE
%-------------------------------------------------------------------------------

\begin{document}


\title{ \textsc{KPF 2018Z}
		\\ %[2.0cm]
		\HRule{0.5pt} \\
		\LARGE \textbf{\uppercase{
		    Sztuczna inteligencja w metafizyce} \\ 
	    \normalsize Próby przeprowadzenia dowodu na istnienie Boga za pomocą komputera }
		\HRule{2pt} \\ [0.5cm]
		%\normalsize \vspace*{5\baselineskip}
		\includegraphics[width=0.5\linewidth]{anzelm.jpg}
		}


\author{Artur M. Brodzki }


\pagenumbering{gobble}
\maketitle
\newpage
\pagenumbering{arabic}

\renewcommand\tablename{Tabela}

\newtheorem{theorem}{Twierdzenie}
\newtheorem{definition}{Definicja}
\newtheorem{axiom}{Aksjomat}
\newtheorem{axiom-g}{Aksjomat}
\newtheorem{definition-g}{Definicja}
\newtheorem{theorem-g}{Twierdzenie}
\newtheorem{lemma}{Lemat}
\newtheorem{corollary}{Fakt}

%-------------------------------------------------------------------------------
% Section title formatting
\sectionfont{\scshape}
%-------------------------------------------------------------------------------

%-------------------------------------------------------------------------------
% BODY
%-------------------------------------------------------------------------------
\begin{multicols}{2}

\section{Wstęp} \label{sec:intro}
Spory o możliwość racjonalnego udowodnienia istnienia Boga (lub bogów) toczą się co najmniej od czasów starożytnych, do dnia dzisiejszego nie znalazły zadowalającego wszystkich rozwiązania -- i nie zanosi się na to, by udało się je zakończyć w przyszłości. Kontrowersje przynajmniej częściowo wynikają z różnych sposobów rozumienia i definiowania samego pojęcia Boga w różnych tradycjach filozoficznych. Starożytnym Grekom znane było pojęcie Absolutu, czyli -- w dużym uproszczeniu -- bytu podstawowego, z którego wszystko inne się wywodzi. Tak rozumiany Absolut bywa utożsamiany z Bogiem w rozumieniu kultury chrześcijańskiej, chociaż zachodzą tutaj istotne różnice -- Absolut jest bowiem bytem całkowicie bezosobowym, natomiast chrześcijański Bóg posiada własną samoświadomość i wchodzi w osobową relację ze światem stworzonym. Obie tradycje starano się łączyć ze sobą w średniowiecznej scholastyce; z tego okresu pochodzi klasyczny zestaw 5 dowodów na istnienie Boga autorstwa św. Tomasza z Akwinu. 

O ile jednak starożytni i średniowieczni autorzy mieli do dyspozycji jedynie siłę swej własnej intuicji i naturalnej inteligencji, to my -- ludzie XXI wieku -- możemy już wykorzystać do analizy problemu inteligencję sztuczną. Szczególnie interesujące wydają się próby przeprowadzenia dowodu na istnienie Boga (nieważne na tę chwilę, w jakiej konkretnej tradycji) za pomocą komputerowych systemów automatycznego dowodzenia. Aby to było możliwe, należy jednak wpierw uściślić i sformalizować samo pojęcie Boga i jego podstawowych własności w języku nowoczesnej matematyki. Okazuje się, że zadanie to zostało wykonane jeszcze w erze przed-komputerowej, przez niemieckiego matematyka i logika Kurta G\"odla. Opierając się na znanym, lecz odrzuconym jeszcze w średniowieczu, dowodzie ontologicznym Anzelma z Canterbury, stworzył on własny dowód zapisany w formalizmie współczesnej logiki modalnej, znany powszechnie jako dowód ontologiczny G\"odla. Jakkolwiek wykazuje on swoje własne problemy, to jego forma jest na tyle zmatematyzowana, że nadaje się on do komputerowej analizy. Rola i status tego dowodu, jak również jego modyfikacje i możliwość uniknięcia problemów, pozostają nadal problemem otwartym. 

W następnych rozdziałach opiszę pokrótce kształt G\"odlowskiego dowodu, wychodząc od -- prostszego do zrozumienia -- dowodu Anzelma. Następnie opiszę próby weryfikacji dowodu za pomocą komputera, a na koniec - możliwe modyfikacje i perspektywy na przyszłość. 

\section{Dowód Anzelma}
Anzelm zaczyna swój dowód od uściślenia samego pojęcia boskości. Najpierw jednak czyni pewne założenia wstępne: 
\begin{axiom} \label{axiom:1}
	Wszystkim istniejącym bytom można przypisać cechę \emph{doskonałości}. Różne byty posiadają cechę doskonałości w róznym stopniu. 
\end{axiom}
\begin{axiom} \label{axiom:2}
	Byt istniejący obiektywnie jest bardziej doskonały, niż identyczny byt, ale istniejący tylko w ludzkim umyśle. 
\end{axiom}
W tak zdefiniowanym aparacie pojęciowym widoczne jest echo średniowiecznego sporu o uniwersalia -- tj. o to, czy byty abstrakcyjne istnieją w obiektywnej rzeczywistości, czy też jedynie w ludzkimu umyśle, jako użyteczne kategorie. Natomiast doskonałość można tutaj rozumieć zarówno w sensie moralnym, jak i jako piękno czy użyteczność. Dokładny sposób wartościowania obiektów pod względem tej cechy nie jest jednak dla Anzelma istotny; mając już bowiem ustalone podstawowe definicje, mógł Anzelm przystąpić do zdefiniowania samego pojęcia Boga:
\begin{definition} \label{def:god}
	Bóg jest to byt, od którego nie ma (wręcz nie można sobie wyobrazić) żadnego bytu bardziej doskonałego. 
\end{definition}
Na bazie tak sformułowanej definicji, daje się już udowodnić twierdzenie:
\begin{theorem} \label{theorem:god}
	Bóg jest bytem istniejącym realnie, poza ludzkim umysłem. 
\end{theorem}
\begin{proof}
	Dowód twierdzenia odbywa się przez zaprzeczenie. Załóżmy, że Bóg istnieje tylko jako wytwór myśli człowieka. Wynika z tego, że nie jest to idea najdoskonalsza ze wszystkich, można bowiem wyobrazić sobie Boga bardziej doskonałego: takiego, który istnieje w realnej rzeczywistości. Wniosek ten jest jednak sprzeczny z przyjętą Definicją \ref{def:god}. Uznając założenie początkowe za prawdziwe, otrzymujemy sprzeczność -- a zatem Bóg musi być bytem istniejącym realnie. 
\end{proof}
Dowód Anzelma spotkał się z krytyką i to niemal natychmiast po opublikowaniu; powrócimy do tego w następnych rozdziałach. Przede wszystkim jednak jest to dowód bardzo nowoczesny w formie i okazuje się, że można go łatwo przełożyć na język współczesnej matematyki. Dokonał tego Kurt G\"odel w 1941 roku, jakkolwiek -- z przyczyn kulturowych, a mianowicie obaw G\"odla o reakcję środowiska naukowego -- prace na ten temat zostały opublikowane dopiero 9 lat po jego śmierci [G\"odel, 1995]. 

\section{Dowód G\"odla}
Pełna postać dowodu G\"odla jest skomplikowana i nie będę jej tutaj szczegółowo przytaczał. Przedstawię jedynie podstawowe aksjomaty, definicje i twierdzenia pośrednie -- dla zobrazowania ogólnej komcepcji i zilustrowania faktu, że treści metafizyczne dają się zapisać w języku dzisiejszej logiki. 

Dowód G\"odla wykorzystuje aparat matematyczny logiki modalnej, należącej do tzw. logik nieklasycznych i będącej w zasadzie rozszerzeniem klasycznego rachunku zdań o dwa dodatkowe spójniki, tzw. spójniki modalne: spójnik możliwości $\diamondsuit p$, czytany jako ,,jest możliwe, że $p$'' oraz spójnik konieczności $\Box p$, czytany jako ,,koniecznie $p$''. Za pomocą logiki modalnej można wyrażać stwierdzenia charakteryzujące się różnym stopniem pewności: \emph{Jutro nie musi padać.} \emph{Możliwe, że ustawa zostanie uchwalona.} \emph{Z pewnością poniesie on tego konsekwencje. } Oba spójniki modalne można sprowadzać do siebie nawzajem, za pomocą przekształceń analogicznych do zwykłych praw de Morgana:
\begin{align*}
\neg \diamondsuit Z & \Leftrightarrow \Box \neg Z \\ 
\neg \Box Z & \Leftrightarrow \diamondsuit \neg Z
\end{align*}
		
G\"odelwykorzystuje logikę modalną do pokazania, że przy dość ogólnym zbiorze założeń wstępnych prawdziwe jest stwierdzenie, że ,,Bóg istnieje w sposób konieczny''. 

Opiszę teraz pokrótce samą postać dowodu. Zakładamy najpierw - podobnie jak Anzelm - że obiekty $x$ posiadają różne cechy, tj. predykaty $\varphi(x), \psi(x), \xi(x)$ w sensie logicznym -- i że te cechy dają się opisać jako ,,pozytywne'' $P(\varphi)$ lub ,,negatywne'' $\neg P(\psi)$. Wprowadzamy również symboliczne oznaczenie cechy boskości $G: G(x)$ oznacza zatem, że $x$ jest Bogiem. Na początek przyjmujemy kilka podstawowych aksjomatów dotyczących cech pozytywnych i negatywnych. 
\begin{axiom-g} \label{axiom:godel1}
	Brak dobra jest zły i vice versa: 
	\begin{align*}
	\neg P(\varphi) & \Leftrightarrow P(\neg \varphi) \\ 
	P(\varphi) & \Leftrightarrow \neg P( \neg \varphi )
	\end{align*}
\end{axiom-g}
\begin{axiom-g} \label{axiom:godel2}
	Z dobra nie może wynikać zło (dobro zawsze implikuje dobro): 
	\begin{equation*}
	\left( P(\varphi) \wedge \Box \forall x: \varphi(x) \Rightarrow \psi(x) \right) \Rightarrow P(\psi)
	\end{equation*}
\end{axiom-g}
\begin{axiom-g} \label{axiom:godel3}
	Dobro jest absolutne (cechy dobre są zawsze, dobre w każdym możliwym świecie):
	\begin{equation*}
	P(\varphi) \Rightarrow \Box P(\varphi)
	\end{equation*}
\end{axiom-g}
Powyższe aksjomaty oddają intuicje dotyczące cech dobrych (pozytywnych) i złych (negatywnych), przyjmowane zazwyczaj mniej lub bardziej świadomie przez większość ludzi. Ich dość ogólny charakter decyduje o sile dowodu ontologicznego, jednak również o jego słabościach, o czym szczegółowo opowiem w następnych rozdziałach. 
Następnie Gödel definiuje pojęcie Boga $G(x)$:
\begin{definition-g} \label{def:godel1}
	Bóg to obiekt posiadający wszystkie cechy pozytywne: 
	\begin{equation*}
	G(x) \Leftrightarrow \forall \varphi \left( P(\varphi) \Leftrightarrow \varphi(x) \right)
	\end{equation*}
\end{definition-g}
Mogłoby się wydawać oczywiste, że predykat $G$ jest pozytywny, jednak -- zaskakująco -- $P(G)$ nie wynika z aksjomatów \ref{axiom:godel1} - \ref{axiom:godel3}. Jest tak dlatego, że $G$ definiuje się poprzez kwantyfikator po predykatach, a zatem $G$ jest predykatem rzędu wyższego o 1 od pozytywnych cech, które z sobie zawiera. Wprowadzamy zatem dodatkowy
\begin{axiom-g} \label{axiom:godel4}
	P(G)
\end{axiom-g}
Z tak zdefiniowanych założeń możemy już wyprowadzić kilka interesujących wyników. Przede wszystkim okazuje się, że dla każdej pozytywnej własności $\varphi$ możemy znaleźć przynajmniej jeden obiekt, który tę własność posiada. Mówimy, że każda pozytywna właściwość jest \say{potencjalnie egzemplifikowana} (ang. \emph{possibly exemplified}). 
\begin{theorem-g} \label{th:godel1}
	$P(\varphi) \Rightarrow \diamondsuit \exists x: \varphi(x)$
\end{theorem-g}
Na podstawie Twierdzenia \ref{th:godel1} możemy wykazać, że istnienie Boga jest faktem możliwym:
\begin{theorem-g} \label{th:godel2}
	$\diamondsuit \exists x: G(x)$
\end{theorem-g}
Zależy nam jednak na pokazaniu, że istnienie Boga jest faktem koniecznym. Potrzebujemy do tego kolejnej definicji. G\"odel był pod wielkim wrażeniem filozofii Leibniza i uwidacznia się to w jego dowodzie ontologicznym. Definiuje on formalnie pojęcie esencji:
\begin{definition-g}
	Predykat $\varphi$ jest \emph{esencją} $x$, gdy wynikają z niego wszystkie własności obiektu $x$:
	\begin{align*}
	& \varphi\ \emph{ess}\ x \Leftrightarrow \varphi(x) \wedge \forall \psi: \\ 
	& \psi(x) \Rightarrow \Box \forall y: \left( \varphi(y) \Rightarrow \psi(y) \right)
	\end{align*}
\end{definition-g}
Czytelnik mający nieco praktyki w logice formalnej może się już domyślać zachodzenia następującego faktu:
\begin{corollary}
	$G(x) \Rightarrow G\ \emph{ess}\ x$
\end{corollary}
Należy teraz sformalizować kluczową części dowodu Anzelma, czyli założenie, że obiekt istniejący realnie jest bardziej doskonały od identycznego obiektu, ale istniejącego tylko w ludzkim umyśle: 
\begin{definition-g}
	Obiekt $x$ istnieje w sposób konieczny $E(x)$, jeśli dla każdej esencji $\psi$ obiektu $x$ istnieje co najmniej jeden obiekt posiadający cechę $\psi$:
	\begin{equation*}
	E(x) \Leftrightarrow \forall \psi: \left( \psi\ \emph{ess}\ x \Rightarrow\Box\ \exists x: \psi(x) \right)
	\end{equation*}
\end{definition-g}
Zgodnie z rozumowaniem Anzelma, wprowadzamy aksjomat, że $E$ jest cechą pozytywną:
\begin{axiom-g}
	P(E)
\end{axiom-g}
Ponieważ $E(x)$ jest cechą pozytywną, a $G$ jest jedyną esencją Boga, to -- w połączeniu z twierdzeniem \ref{th:godel1} -- uzyskujemy natychmiastowy wniosek:
\begin{theorem-g} \label{th:goedel3}
	$\Box\ \exists x: G(x)$
\end{theorem-g}
Udowodniliśmy, że Bóg istnieje w świecie w sposób konieczny. 

\section{Komputerowa analiza dowodu}
Dowód G\"odla korzysta z nieklasycznej logiki modalnej, i to logiki modalnej wyższego rzędu -- ponieważ wykorzystuje predykaty drugiego rzędu i kwantyfikatory po predykatach (m.in. aksjomat \ref{axiom:godel1}, \ref{axiom:godel3}, \ref{axiom:godel4}). Logiki wyższego rzędu są trudne do komputerowej analizy, ponieważ problemy wyrażone w takich logikach są w ogólności nieobliczalne; dodatkową trudność stanowi reprezentacja spójników modalnych $\diamondsuit$ i $\Box$. Niemniej okazuje się, że logiki modalne wyższego rzędu (HOML, ang. \textit{Higher-Order Modal Logic}) dają się sprowadzić do zwykłej, niemodalnej logiki wyższego rzędu (HOL, ang. \textit{Higher-Order Logic}) poprzez ominięcie spójnika $\Box$ i kilka innych operacji dot. semantyki [Benzm\"uller, 2014]. Tak uproszczona postać dowodu okazuje się być matematycznie równoważna, a co więcej - nadaje się już do zautomatyzowanej analizy. Pierwsze interesujące rezultaty udało się otrzymać w 2014 roku, przy użyciu znanych od dawna programów wspomagających dowodzenie: Isabelle, LEO-II, Satallax i Coq. Udało się potwierdzić następujące fakty:
\begin{itemize}
	\item Zbiór aksjomatów \ref{axiom:godel1} -- \ref{axiom:godel4} jest niesprzeczny. 
	\item Twierdzenie \ref{th:goedel3} jest dowodliwe na bazie przyjętych założeń\footnote{Wymienionym programom nie udało się jednak wytworzyć kompletnego dowodu w formie jawnej, a jedynie stwierdzić, że jest to możliwe. }, tym samym dowód G\"odla jest -- formalnie rzecz biorąc -- poprawny.
	\item Istnieje tylko jeden Bóg spełniający przyjęte założenia -- na  gruncie przyjętych aksjomatów można więc udowodnić prawdziwość monoteizmu. 
\end{itemize}

Dowód G\"odla cierpi jednak na swoje własne problemy, częściowo odziedziczone po dowodzie Anzelma. Zarzut podniesiony jeszcze w średniowieczu sprowadza się do tego, że wykorzystując zaproponowaną przez Anzelma konstrukcję myślową można udowodnić istnienie bardzo wielu bytów, np. idealnej wyspy (taka wyspa musiałaby wszak istnieć w rzeczywistości, inaczej nie byłaby idealna), jak również licznych pół-bogów, czy -- odwracając wartościowanie -- demonów i innych, niekoniecznie pożądanych przez nas jako filozofów, bytów. G\"odel znał te zarzuty i projektując swój dowód starał się uniknąć podobnej pułapki -- jednak w oryginalnej wersji dowodu występuje podobne, choć bardziej wyrafinowane zjawisko modalnego kolapsu: wszystko, co jest możliwe, jest również konieczne. Istnienie tego problemu było podnoszone już w latach 80-tych [Sobel, 1987], a całkiem niedawno jego występowanie zostało potwierdzone analizą komputerową [Benzm\"uller, 2014]. 

Podejmowano próby modyfikacji dowodu w celu uniknięcia modalnego kolapsu poprzez osłabienie aksjomatu \ref{axiom:1} [Anderson, 1990]. W jego oryginalnej postaci, ,,pozytywność'' i ,,negatywność'' są swoimi wzajemnymi zaprzeczeniami i niemożliwe są predykaty klasyfikowane jako ,,neutralne''. Zamiana równoważności na implikację:
\begin{equation*}
P(\varphi) \Rightarrow \neg P( \neg \varphi )
\end{equation*}
dopuszcza istnienie predykatów neutralnych i pozwala na uniknięcie kolapsu [Anderson, 1996]. Tak zmodyfikowany zestaw aksjomatów po jego weryfikacji przez oprogramowanie okazał się jednak niespójny [Benzm\"uller, 2014], [Benzm\"uller, 2016], co było nowym i dosyć zaskakującym rezultatem. Poszukiwania takiej postaci dowodu, która pozwoliłaby uniknąć powyższych problemów, pozostają zatem nadal problemem otwartym. 

\section{Podsumowanie} \label{sec:summary}
Spór o istnienie Boga i możliwość udowodnienia tego faktu toczy się w filozofii od czasów starożytnych. Nowoczesnym dowodem tego rodzaju jest dowód ontologiczny G\"odla, oparty na klasycznym dowodzie ontologicznym Anzelma, i sformalizowany w języku współczesnej logiki. Dzięki wykorzystaniu nowoczesnego oprogramowania, istnieje możliwość analizy tego rodzaju dowodów w sposób zautomatyzowany. Pozwala to rozwiązać niektóre znane problemy, jednocześnie stawia przez nami nowe wyzwania i pytania, na które wciąż nie znamy odpowiedzi. 


%-------------------------------------------------------------------------------
% REFERENCES
%-------------------------------------------------------------------------------
\section*{Literaura}

\noindent [Anderson, 1990] Anderson, C. A. (1990). Some emendations of G\"odel’s ontological proof. \textit{Faith and Philosophy}, str. 291–303. Dostęp zdalny (PDF): \url{https://appearedtoblogly.files.wordpress.com/2011/05/anderson-anthony-c-22\\some-emendations-of-gc3b6dels-\\ontological-proof22.pdf} (aby link był poprawny, należy usunąć z niego oba ukośniki ,,\textbackslash''). 
\\ \\
\noindent [Anderson, 1996] Anderson, Curtis Anthony; Geettings, M. (1996). G\"odel’s ontological proof revisited. \textit{Lecture Notes in Logic}, str. 167–172. Dostęp zdalny (PDF): \url{https://projecteuclid.org/download/pdf_1/euclid.lnl/1235417020}
\\ \\
\noindent [Benzm\"uller, 2014] Benzm\"uller, Ch; Paleo, Bruno W. (2014). Automating G\"odel’s Ontological Proof of God’s Existence with Higher-order Automated Theorem Provers. \textit{European Conference on Artificial Intelligence.} IOS Press, str. 93-98. Dostęp zdalny (PDF): \url{http://page.mi.fu-berlin.de/cbenzmueller/papers/C40.pdf}
\\ \\
\noindent [Benzm\"uller, 2016] Benzm\"uller, Ch; Paleo, Bruno W. (2016). The Inconsistency in G\"odel's Ontological Argument: A Success Story for AI in Metaphysics. \textit{International Joint Conference on Artificial Intelligence.} AAAI Press. str. 936. Dostęp zdalny (PDF): \url{https://www.ijcai.org/Proceedings/16/Papers/137.pdf}
\\ \\ 
\noindent [G\"odel, 1995] G\"odel, K. (1995). Texts relating to the ontological proof. W: \textit{Unpublished Essays and Lectures}, str. 429–437. Oxford University Press. 
\\ \\ 
\noindent [Sobel, 1987] Sobel, J. H. (1987). G\"odel’s ontological proof. W: \textit{On Being and Saying: Essays for Richard Catwright}. str. 241–261. MIT Press, Oxford/MA
\\ \\ 
\noindent [Sobel, 2004] Sobel, J. H. (2004). Logic and Theism: Arguments for and Against Beliefs in God. Cambridge U. Press, 2004.

\end{multicols}
\end{document}
